\section*{Convalescence}
\addcontentsline{toc}{section}{Convalescence}

Researchers and
organizations spent many man hours fighting the worm. Time was lost rebooting
machines and jobs that depended on infected machines were stalled for days.
Email communication also took a hit in cases where networks were shutdown to
quarantine the spread of the infection\cite{seeley_tour_1989}. For several days,
a large portion of the internet and its connected computers were unuseable.

This opportunity cost the Morris Worm crisis caused is rather shallow, however,
when compared to the reaction that the worm illicited. The phase in which one
recovers after illness caused by an infectious disease is called \textit{convalescence}. This
section addresses the recovery of the computing community and the
general public through their reponses to the worm crisis.

\subsection*{Awareness and Perception}
\addcontentsline{toc}{subsection}{Awareness and Perception}
The Morris Worm crisis elevated awareness of computing, the internet, and
security. Prior to the crisis, much of the general public didn't even
know about the internet, and therefore could not fathom the impact that a
cyber attack could have. The computing community, aware of some of the very
vulnerabilities that the worm exploited, knew that someone one \textit{could}
set off a catastrophic cyber attack, but didn't suspect that anyone
\textit{would}. The computer literate and computer agnostic were both faced
with questions regarding the ethical nature of what Robert Morris had done.

\subsubsection*{Public}
As stories about the worm crisis began to hit the news, people were suddently
confronted with new terminology. ``Computer Virus'', and ``Internet'', amongst
others, became new common vocabulary words for people who hadn't heard them
before. The Morris Worm had thrust the concepts of large communications networks
linking computers and attacks against them into the spotlight. People
began digesting the new ideas, at times rather comically.
\cite{lee_washpost_2013} quotes, from an
interview with Dr. Spafford:
\begin{quote}
``I got one call from a newspaper in Southern Indiana,'' Spafford says. ``The
reporter asked me, in all earnestness, `Do our readers need to worry about
catching this virus?'''
\end{quote}
\cite{marsan_morris_2008} quotes, from an
interview with Steve Bellovin, a Computer Science Professor at Columbia
University:
\begin{quote}
``The Morris worm was the first time most people ever heard the word 'Internet,'''
Bellovin says. ``For most people, it was a novelty, a strange and wondrous
world \ldots and one rogue operator could take it down. Nobody had ever heard
of the Internet unless you were a computer scientist.''
\end{quote}
The news coverage over the next several months would reinforce a growing public
awareness of two new ideas: that a huge communications network of connected
computers exists and it can be used to perpertrate massive and destructive
cyber attacks.
It is curious to ponder that the existence of the internet and its
potential for misuse both dawned on the public at the same time.
           	
\subsubsection*{Researchers}
Researchers were faced with some realizations of their own. While aware of the
possibility of security breaches, computer scientists didn't appreciate the
magnitude of the damage that could be caused or the likelihood that someone
would actually launch a massive attack. \cite{seeley_tour_1989} points out that
the vulnerabilities exploited by the Morris worm were already known to the
computing community. In fact, Morris himself had warned people about them.
\cite{spafford_internet_1989} relates that ``\ldots at a recent meeting, 
Professor Rick Rashid of Carnegie-Mellon University was heard to claim that
Robert T . Morris, the alleged author of the Worm, had revealed the fingerd
bug to system administrative staff at CMU well over a year ago.'' Aware of many
existing security vulnerabilities, system administrators still considered
their risk for a cyber attack to be miniscule. The Morris Worm showed
them otherwise. 

Even amongst the computer literate, many people didn't know how to classify
the attack without first developing a heightened awareness of cyber attack
taxonomy. Of the flurry of publications generated in response to the crisis,
many of them started by addressing this issue. \cite{spafford_internet_1989-1}
points out:
\begin{quote}
There seems to be considerable variation in the names applied to the program
described here. Many people have used the term worm instead of virus based on its
behavior. Members of the press have used the term virus, possibly because their 
experience  to date has been only with that form of security problem. This usage
has been reinforced by quotes from computer managers and programmers also
unfamiliar with the difference.
\end{quote}
\cite{eichin_microscope_1989} obstinately asserted that the internet
attack was caused by a ``virus''
\footnote{In particular,
\cite{eichin_microscope_1989} thought the Morris Worm should be classified as a
``lysogenetic'' virus based on its behavior with respect to its host.}
. The rest of the research community classified it as a ``worm'', stating that
viruses instert their code into other programs and are passively executed when
the host program runs while worms are standalone programs and use system
resources to actively spread themselves \cite{seeley_tour_1989,
spafford_internet_1989, spafford_internet_1989-1}. The Morris Worm 
incited this new appreciation for semantic nuance in cyber attack
classification. 

Upon inspection of the Morris Worm, researchers realized that the crisis could 
have been much worse. The worm could have deliberately performed malicious
actions on infected machines, like deleting files and trafficking passwords. It
could have been used as a mechanism to spread viruses. It could have installed
back doors or trojan horses. Finally, it could have done all of this more
covertly, making it much harder to discover, diagnose, and destroy.
\cite{spafford_crisis_1989} points out that
\begin{quote}
Entire businesses are now dependent, wisely or not, on the undisturbed
functioning of computers. Many people's careers, property, and lives may be
placed in jeopardy by acts of computer sabotage and mischief.
\end{quote} 

In light of all of this, the computing community realized that the internet at
large needed a more robust immune system.
           	
\subsubsection*{Morris: Criminal or Genius?}
In the wake of the Morris Worm's destructive campaign, an important question
faced the general public and the research community: How should Morris' actions
be regarded? People began to debate about whether Morris was a criminal. There
was also discussion about whether he was a genius, or ``computer whiz''. Faced
with the first major cyber attack of this nature, researchers and the public
developed a broad spectrum of contrasting opinions about Morris and his
behavior.

An immediate question that generated diverse opinions concerned the ethical
nature of Morris' actions, and, by extension, of cyber attack perpetrators in
general. While some felt he was a vicious criminal that should be
punished as harshly as possible, others argued that he was a hero for
demonstrating security flaws. \cite{spafford_crisis_1989} observes that 
\begin{quote}
One oft- expressed opinion, especially by those individuals who believe the
worm release was an accident or an unfortunate experiment, is that the author
should not be punished. Some have gone so far as to say that the author should
be rewarded\ldots
The other extreme school of thought holds that the author should be severely
punished, including a term in a federal penitentiary.
\end{quote}

Ultimately, a middle and more grounded opinion prevailed. The Cornell Commision
that investigated Morris' actions as a Cornell student determined that 
\begin{quote}
Sentiment among the computer science professional community appears to favor
strong disciplinary measures for perpetrators of acts of this kind. Such
disciplinary measures, however, should not be so stern as to damage permanently
the perpetrator's career.
\end{quote}

This sentiment is echoed in \cite{spafford_crisis_1989}:
\begin{quote}
\ldots it would not serve us well to overreact to this particular incident.
However, neither should we dismiss it as something of no consequence\ldots
Furthermore, we should be wary of setting dangerous precedent for this kind of
behavior. Excusing acts of computer vandalism simply because the authors claim
there was no intent to cause damage will do little to discourage repeat
offenses, and may, in fact, encourage new incidents.
\end{quote}

Another question of interest brought to light by the worm crisis concerned the
intellectual prowess associated with cyber attacks. While the public often
envisions computer hackers as geniuses who flex an advanced intellect and
intimate knowledge of computer science to perpetrate cunning attacks, the
academic community was quick to point the fallacy in such thinking.

            	\begin{itemize}
            	\item Genius computer whiz?
            	\begin{itemize}
            		\item researchers: NO	
            		\item public: Yes	
            	\end{itemize}
       			\end{itemize}


\subsubsection*{Accountability}
Who is at fault: administrators vs. vendors\ldots beginning of
``blame the victim'' mentality

\subsection*{Antibodies}
\addcontentsline{toc}{subsection}{Antibodies}
	\begin{itemize}
	\item prosecution under CFAA
    	\begin{itemize}
    	\item Morris' claim that it was unintentional... "knowingly" vs "intentionally"	.. accident
    	\item Morris' claim that he didn't gain ungranted access because he had access to Cornell computers
    	\item Precedent: the internet falls under CFAA
        \item Question: what if worm hadn't infected government computer?
    	\end{itemize}
	\item computer security as a legitimate field
        	\begin{itemize}
        	\item tons of publications
            \item questions about security design principles (like least privilege)
            \item questions about ethics and law ***(transition to next section)
			\item better security practices 
				\begin{itemize}
				\item need for better passwords
				\item better patching of vulnerabilites
        	    \end{itemize}
        	\end{itemize}
    \item crisis response protocol
    	\begin{itemize}
    	\item formation of CERT at CMU
    	\item phage mailing list
    	\end{itemize}
    \end{itemize}
\section*{Evolution}
\addcontentsline{toc}{section}{Evolution}
In the field of biology, evolution is defined as change in a gene pool over
generational time. Amongst other things, evolutionary biologists study this
gradual change, tracing inhertited genetic traits from ancestors on to
their descendants. Through their exhibition of similar traits, several
other worm attacks that have occured in the past several decades appear to be
descendants of the Morris Worm. As the cyber attacks evolved, so
too did awareness and perception of cyber security. This section explores this
slow evolution of worm attacks and cyber security awareness.

\subsection*{Descendants}
\addcontentsline{toc}{subsection}{Descendants}
The Morris Worm provides concrete, ancestral examples of several commonly used
attack vectors in computer security, most notably:
\begin{itemize}
  \item using email as a spreading mechanism,
  \item identifying targets for infection by reading connection lists maintained
  on an infected machine,
  \item discovering new targets for infection by randomly scanning a network,
  \item exploiting widely known buffer overflow and trapdoor vulnerabilities
  \item and attacking weak passwords.
\end{itemize}
Several subsequent worms have found new ways to capitalize on some of these same
exploits.
\subsubsection*{Code Red}
In the summer of 2001, the Code Red Worm infected about 500,000 servers.
Fisher\cite{fisher_living_2002} calls
the worm ``a direct descendant of Morris' creation''.
It spreads by exploiting a known buffer overflow vulnerability in Microsoft
Internet Information Services (IIS) servers. When the first infection was
discovered, a patch for the IIS vulnerability had already existed for a
month\cite{fisher_code_2001}. After infecting a new host, the worm scans the
internet for other hosts running the same IIS server to infect. 

Much like the Morris Worm, the Code Red Worm contains several telling bugs,
including a bug in the random IP address generator that it uses to scan the internet.
Berghel\cite{berghel_code_2001} points out
\begin{quote}
An interesting byproduct of the bug in the original Code Red was that instead 
of creating random paths to each infected server, Code Red infected each new
server via the same path as its predecessor, thereby leaving a log of infected
servers behind with each new infection. A failure of epidemic proportions in
hiding one's tracks (should the perpetrator be identified, perhaps we should
give him or her an honorary ``F'' in software design).
\end{quote}

Unlike the Morris Worm, the Code Red Worm did more than just reproduce. It also
contained code to display mischevious messages in web browsers on infected
hosts, and even to launch a distributed denial of service attack (DDOS) on the White
House website on July 19$^{th}$ 2001. Fortunately, researchers found that
the code in the worm designed to flood the White House website with traffic used
the site's actual IP address, rather than the URL ``Whitehouse.gov''. The planned
attack was therefore easily thwarted by changing the IP
address\cite{berghel_code_2001}.

The attack incited a familiar discussion about who is to blame for security
vulnerabilites. Berghel\cite{berghel_code_2001} relates:
\begin{quote}
Many feel that the responsibility belongs to the administrators who failed to
heed repeated warnings to patch their systems\ldots.
Others believe the real culprits are software vendors that release unsecure
products and government agencies that are slow to react to major security threats.
\end{quote}

Notice that blame for the attack was cast upon almost everyone
except the actual creator of the Code Red Worm.

\subsubsection*{Slammer}
The Slammer Worm was released on January $25^{th}$, 2003. Within 10 minutes of
its release it had infected 75,000 machines. It caused
flight cancellations and disabled ATMs. As with previous worms, Slammer
exploited a buffer overflow vulnerability in Microsoft SQL Server 2000. As with
previous worms, the vulnerability that Slammer exploited was well known. A
patch had even been released 6 months prior to the worm's release, but many
system administrators hadn't applied it. It scanned random IP addresses on the
web to discover new victim machines to
infect\cite{tynan_dawn_2003}\cite{panko_slammer:_2003}.

The finger of blame was again pointed at almost everyone but
its creator. Schultz\cite{schultz_security_2003} notes that ``Microsoft will
undoubtedly continue to hold users at fault for not installing
appropriate patches, but the real problem is the
bug-filled software that Microsoft and other
vendors have continually produced over the
years.'' 

\subsubsection*{Blaster}
The Blaster Worm was first discovered on August $11^{th}$, 2003. Within a week
it spread to over 100,000 Microsoft Windows systems. The Blaster worm randomly
scans the internet for new victims. When a connection is found, it exploits a
known trap door vulnerability in a Microsoft server for which there is a
patch. It contains code to launch a DDOS attack on Microsoft's Windowsupdate.com
website\cite{bailey_blaster_2005}. 

As with previous worms, it contains bugs and other poor coding practices that
reveal fundamental programming deficiencies of its creator. 
Bailey\cite{bailey_blaster_2005} points out that its repeated and redundant
re-seeding of the random number generator that it uses ``indicates that the worm author
significantly lacks understanding of random number generators.''

\subsubsection*{Conficker}
The Conficker Worm began to spread in 2008 and infected millions of Windows
computers. It does do something new: After
infecting a host, it opens a communication channel and awaits further
instructions, effectively turning the host machine into a ``zombie''. 
Lee\cite{lee_washpost_2013} suggests that its authors have ``finally achieved
something like Morris's original, unrealized vision of using a worm to create a
vast network of computers operating under the control of the malware's author.''
A network of these zombie computers is known as a ``botnet'' and ``unscrupulous
individuals use them to send spam e-mail messages, overwhelm Web sites with
traffic, or perform other nefarious tasks.''\cite{lee_washpost_2013}
		
\subsection*{Evolved Awareness and Perception}
\addcontentsline{toc}{subsection}{Evolved Awareness and Perception}

In the decades after the Morris Worm crisis, computers have become ubiquitous.
As a result, people have generally become more computer literate. Unfortunately,
this hasn't necessarily led to improved awareness of computer security and many of the
issues brought to light by the Morris Worm incident are just as relevant now as
they were in 1988. As demonstrated in previous sections, new cyber attacks
written by sloppy programmers continue to exploit old vulnerabilities that
simply haven't been patched yet.

The victim shaming mentality seen after the Morris Worm continues
to prevail as vendors blame administrators for not applying patches and
updates, administrators blame users for using weak passwords, and users blame
vendors for releasing vulnerable software, all seeming to forget about the
author of the cyber attack. Perhaps this is because the the internet is no longer
thought of as a friendly place like it was in 1988 before the Morris Worm was
unleashed. Instead, people now generally assume that attackers are out there
waiting to steal our information and compromise our systems.
These attackers are less often regarded as brilliant and more often
regarded as despicable frauds out to steal financial information. As a
result, society expects that system administrators and personal computer users
should employ every conceivable security measure. The increased pressure makes
it that much easier to blame victims, rather than attackers, when security
breaches occur.

The CFAA is still being expanded to deal with new types of cyber crime
\cite{adams_controlling_1996}. The CERT also continues to grow and respond to
computer security crises of all shapes and sizes. While it handled just a few
incidents per year after its creation, by 1993 it was handling thousands of
incidents\cite{fithen_cert_1994}. Furthermore, many new crisis response teams
have been established. The procedure to set up and maintain a Computer
Security Incident Response Team (CSIRT) has been well studied and formalized
and many organizations employ their own internal CSIRTs. 

As Lee\cite{lee_washpost_2013} sums up:
\begin{quote}
Today, the Internet is infested with malware that works a lot like the software
Morris set out to build a quarter-century ago. And the community of Internet
security professionals who fight these infections can trace the roots of their
profession back to the events of November 1988.
\end{quote}

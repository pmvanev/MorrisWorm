\section*{Incubation}
\addcontentsline{toc}{section}{Incubation}
\setcounter{page}{1}

The Internet's massive size and popularity today make it hard to imagine its
humble beginning. What we now know as the Internet began as a research project
sponsored by the Defense Advanced Research Project Agency (DARPA) called the
Arpanet. On the day of its birth in December, 1969, the Arpanet consisted of
only four connected nodes: UCLA, SRI, UCSB, and the University of Utah. Lines
of communication between these nodes were ``two million times slower than
today's fastest networks'' \cite{strawn_masterminds_2014}. 

The Arpanet opened the door for a flurry of new research projects. New ideas
grew into numerous applications like file sharing, remote logon, and email. Soon, more
nodes and networks were added. By the mid 1980s, the growth and success
of the Arpanet prompted the National Science Foundation to build a network to
connect research universities to some of its newly built supercomputer centers. 
This new network came to be known as the NSFnet.
It made internetworking ubiquitous amongst universities and researchers,
eventually connecting ``more than 2,000 universities and colleges and a
number of high tech companies'' \cite{strawn_masterminds_2014}. Ultimately, the
1990s would see the NSFnet becoming commercialized into the Internet that we
know today, but not without some growing pains. This paper focuses on one of the
earliest and most vehement growing pains the developing internet\footnote
{
The above summary of the
birth of the Internet used the names ``Arpanet'', ``NSFnet'', and ``Internet''
for various stages in a growth of a widely connected network of computers. For
simplicity, the rest of this paper will collectively refer to all of these
stages as the ``internet''.
} 
would face,
known initially as the ``Internet Worm'' and later as the ``Morris Worm''.


An important detail in the internet's conception and early life is the
attitude of its creators.
This internet was developed by researchers focused on
optimizing its efficiency, not its security. Some security
measures were built in as an afterthought, but security was not woven into
the core fabric of the internet or its applications. As the internet
grew, researchers who used and designed
it were generally assumed to have good intentions. According to
\cite{lee_washpost_2013}, ``\ldots the Internet was like a small town where
people thought little of leaving their doors unlocked. Internet security was
seen as a mostly theoretical problem, and software vendors treated security
flaws as a low priority.'' The same article quotes the sentiments of Dr. Eugene
Spafford \footnote{
Dr. Spafford is a computer security researcher and professor at Purdue
University. He is also executive director of the Purdue's Center for Education
and Research in Information Assurance and Security and an internationally
recognized expert in computer security.}:

\begin{quote}
The majority of people had some tie to computation for their jobs. I wouldn't
say that we trusted each other, but there was more a community sense of caring
for the stability and appropriate use of the computing systems\ldots \\
There was no such thing as a firewall back then. You didn't have people who were
vandals or anarchists or criminals as much. 
\end{quote}

Thus, there was no perceived need for strong security. This very lack of
security would ultimately provide a temperate environment for the incubation of
a virulent infection: the Morris Worm.

The remainder of this
paper tunnels through the worm hole in history that the Morris Worm
has left behind. First, it covers \textit{what} happened during the worm's
outbreak and eradication. Second, it recounts the discoveries that were made
by researchers about \textit{how} the worm works upon its dissection. Third,
it discusses the \textit{immediate impact} that the worm had on the research
community and the general public as they convalesced from the infection. Fourth,
it traces the worm's \textit{long term influence} on the evolution of future
worms and shifting cyber-security perspectives. Finally, the paper
provides a post mortem synthesis on the quality and depth of the worm's overall
impact on the history of computer security. 

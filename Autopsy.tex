\section*{Autopsy}
\addcontentsline{toc}{section}{Autopsy}
In the wake of the Morris Worm outbreak, several
publications presented in depth analyses of the worms
anatomy based on disassembled worm code. This section explores the worm's
anatomy as presented by these dissections at relatively shallow depth. Readers interested
in more detail are encoured to read the following publications on which this
section is based: 
\cite{seeley_tour_1989}
\cite{spafford_crisis_1989}
\cite{spafford_internet_1989}
\cite{eichin_microscope_1989}

The Morris Worm is written in C. Its head consists of a 99 line ``bootstrap''
or ``vector'' program
\footnote{
also called a ``grappling hook'', or the ``l1'' program because it is found in 
the file \textit{l1.c} \cite{spafford_crisis_1989}
}
, used to download the main body of the worm and get it up
and running on a new machine. About 3200 lines of C code comprise the worm's
body. It contains code to help the worm preserve itself, discover and infect
new machines, crack user passwords, and spread. It also contains unused code
and storage space, as well as
numerous software bugs and mistakes.


\subsection*{Self Preservation}
\addcontentsline{toc}{subsection}{Self Preservation}
The worm comes equipped with several mechanisms disguise itself and even fend off potential attacks form system administrators. Upon infection, a
new worm reads any necessary files into memory and deletes them so that it
leaves no trace on disk before zeroing out its argument list. It then changes
its name to \textit{sh} and masquerades as a harmless shell process. To
continue to camouflage itself throughout its lifecycle, the worm occassionally uses the
UNIX \textit{fork} system call to change its process I.D. This also renews the
long running worm's c.p.u. allocation priority in the eyes of the UNIX
scheduler, which may slowly reduce the amount of c.p.u. time granted to long
running processes. All of these tactics allow the worm to blend in and run
unnoticed, but hiding is just one of its defense mechanisms.

The worm also employs measures to obscure itself in case of its discovery. A
basic form of encryption based on the logical exclusive or operation is used to
encode disk files it uses. The worm disables core files to protect against
accidental core dumps in the event of a mistake, or even forced core dumps by a
prodding system administrator. Unnecessary symbols are removed from the worm
program symbol table to make analysis more difficult in the event that a worm is
captured. These hurdles serve to passively obstruct worm analysis, but the worm
takes its own security a step further.

The Morris Worm implements its own active authentication.
When the bootstrap program downloads the worm body to a newly infected host, the client
program authenticates the worm server via the exchange of a ``magic number'', a
random number that both parties agree upon prior to communication. 
This prevents anyone from battling the worm with imposter worms. Equipped with
tools to hide on a host system, change its appearance, and actively authenticate
other worms, a worm program is reasonably capable of fending for itself.  With
multiple methods of self preservation at its disposal, the worm can focus on
its other work.

\subsection*{Password Cracking}
Once it has infiltrated a system, the worm opens the password file
containing hashed account passwords and begins to try to crack them, guessing
different passwords, hashing them, and comparing the results to the entries in
the password file. The worm generates guesses for user passwords in the
following ways:
\begin{itemize}
\item Users and administrators sometimes carelessly fail to change
default passwords, or even to enter a password for a new account. This means
that trying a blank password, or common things \textit{admin} often allow access
to an account.
\item To remember passwords more easily, users often choose simple, easy to
guess passwords. The worm tries very simple passwords, like the user name or the
user name backwards, in the hopes that it can easily ``guess'' a user's
password. The worm also carries with it a miniature dictionary of
known commonly used passwords to try.
\item For convenience, users often use the same password on multiple
machines. If this is the case, once the worm has a user's password on one
machine, it is able to execute remote shells on neighboring machines with ease.
\item Finally, if none of the previous methods work, the worm begins to try all
of the words in the UNIX online dictionary, both capital and lowercase.
\end{itemize}
Notice that none of these methods attempt to attack the hashing algorithm used
to encode the passwords in the password file. Rather, the worm attacks a known
tendency of users to choose weak passwords.


\subsection*{Reproduction}
\addcontentsline{toc}{subsection}{Reproduction}
Because of the security flaws they exploit, the Morris Worm's spreading
mechanisms are particularly noteworthy. It first tries to capitalize on weak
user passwords to create a remote shell with the \textit{rsh} and \textit{rexec}
commands. If this doesn't work, it uses a buffer overrun in the \textit{finger}
application to create a remote shell on a new machine. Finally, if the two
previous methods are ineffective, the worm sneaks in through a trap door in the
\textit{sendmail} application. These exploits are describe in more detail below. 
\subsubsection*{rsh/rexec}
\addcontentsline{toc}{subsubsection}{rsh/rexec}
These programs are intended to allow users to spawn shell sessions on remote
machines. After infiltrating a user account as described in the previous
section the worm can use the \textit{rsh} and \textit{rexec} programs to log
into a remote machine and infect it. The worm seeks out new host machines to
infect by looking for connected host accounts in
system tables and user files, or even scanning the network for other machines.

\subsubsection*{finger}
\addcontentsline{toc}{subsubsection}{finger}
The finger application gives directory information about users. In 1988 it used
the C library \textit{gets} function to read an incoming request into a buffer.
Even in 1988, this function was know to have a classic buffer overflow vulnerability, but
programmers often prioritize ease of use and convention over security when
writing code. The Morris worm exploited this buffer overflow vulnerability,
writing VAX machine code to create a remote shell into to the buffer and then
overwriting the programs stack frame so that the remote c.p.u. would execute
this code when the request finished processing. This allowed the worm to spread
to any VAX machine running the Berkeley version of the finger server.

\subsubsection*{sendmail}
\addcontentsline{toc}{subsubsection}{sendmail}
The sendmail application provides SMTP email service. The Morris Worm exploits
a trap door that is intended to be used for debugging. Entering a debug command
allows one to send commands instead of email to a remote host. The worm uses
this little known feature in an unintended way, sending an anonymous debug email
to a remote host with a copy of the bootstrap program and commands to compile and run
it. This way, the infecting worm doesn't need to open a shell on the remote host
to download and run the bootstrap program.


\subsection*{Maladaptive Traits}
\addcontentsline{toc}{subsection}{Maladaptive Traits}
    	\begin{itemize}
 	    \item detecting itself / coinflip for self termination and immortality bug
        \item no checks for good return values
        \item pass struct instead of pointer to struct
        \item joke about bug fixes in worm
        \item other stuff pointed out by Spaf
        \end{itemize}

\subsection*{Vestigial Organs}
\addcontentsline{toc}{subsection}{Vestigial Organs}
    	\begin{itemize}
        \item UDP to ernie over TCP connection, never sent
        \item accidental leftover files in /tmp (of compilation too slow?)
        \item unused functions and dead code
        \item unused file structure entries
    	\item unused features- could have been much worse... ***(transition into
    next section)
        \end{itemize}


\section*{Autopsy}
\addcontentsline{toc}{section}{Autopsy}
In the wake of the Morris Worm outbreak, several
publications presented in depth analyses of the worms
anatomy based on disassembled worm code. This section explores the worm's
anatomy as presented by these dissections at relatively shallow depth. Readers interested
in more detail are encoured to read the following publications on which the
entirety of this section is based: 
\cite{seeley_tour_1989}
\cite{spafford_crisis_1989}
\cite{spafford_internet_1989}
\cite{eichin_microscope_1989}

The Morris Worm is written in C. Its head consists of a 99 line ``bootstrap''
or ``vector'' program
\footnote{
also called a ``grappling hook'', or the ``l1'' program because it is found in 
the file \textit{l1.c} \cite{spafford_crisis_1989}
}
, used to download the main body of the worm and get it up
and running on a new machine. About 3200 lines of C code comprise the worm's
body. It contains code to help the worm preserve itself, discover and infect
new machines, crack user passwords, and spread. It also contains unused code
and storage space, as well as
numerous software bugs and mistakes.


\subsection*{Self Preservation}
\addcontentsline{toc}{subsection}{Self Preservation}
The worm comes equipped with several mechanisms disguise itself and even fend
off potential attacks from system administrators. Upon infection, a new worm
reads any necessary files into memory and deletes them so that it leaves no
trace on disk before zeroing out its argument list. It then changes its name to
\textit{sh} and masquerades as a harmless shell process. To continue to
camouflage itself throughout its lifecycle, the worm occassionally uses the
UNIX \textit{fork} system call to change its process I.D. This also renews the
long running worm's cpu allocation priority in the eyes of the UNIX
scheduler, which may slowly reduce the amount of cpu time granted to long
running processes. All of these tactics allow the worm to blend in and run
unnoticed, but hiding is just one of its defense mechanisms.

The worm also employs measures to obscure itself in case of its discovery. A
basic form of encryption based on the logical exclusive or operation
encodes its disk files. The worm disables core files to protect against
accidental core dumps in the event of a mistake, or even forced core dumps by a
prodding system administrator. Unnecessary symbols are removed from the worm
program symbol table to make analysis more difficult in the event that a worm is
captured. These hurdles serve to passively obstruct worm analysis, but the worm
takes its own security a step further.

The Morris Worm implements its own active authentication.
When the bootstrap program downloads the worm body to a newly infected host, the client
program authenticates the worm server via the exchange of a ``magic number'', a
random number that both parties agree upon prior to communication. 
This prevents anyone from battling the worm with imposter worms. Equipped with
tools to hide on a host system, change its appearance, and actively authenticate
other worms, a worm program is reasonably capable of fending for itself.  With
multiple methods of self preservation at its disposal, the worm can focus on
its other work.

\subsection*{Password Cracking}
Once it has infiltrated a system, the worm opens the password file
containing hashed account passwords and begins to try to crack them, guessing
different passwords, hashing them, and comparing the results to the entries in
the password file. The worm generates guesses for user passwords in the
following ways:

\begin{itemize}
\item Users and administrators sometimes carelessly fail to change
default passwords, or even to enter a password for a new account. The worm is
therefore able to infiltrate some accounts with a blank password or other
default passwords like \textit{admin}.
\item To remember passwords more easily, users often choose simple, easy to
guess passwords. The worm tries very simple passwords, like the user name or the
user name backwards, in the hopes that it can easily guess a user's
password. The worm also carries with it a miniature dictionary of
known commonly used passwords to try. 
\item If none of the previous methods work, the worm begins to try all
of the words in the UNIX online dictionary, both capital and lowercase.
\item For convenience, users often use the same password on multiple
machines. If this is the case, once the worm has a user's password on one
machine, it tries to login as the user with the same password on neighboring
machines.
\end{itemize}

Notice that none of these methods attempt to attack the hashing algorithm used
to encode the passwords in the password file. Rather, the worm attacks a known
tendency of users to choose weak passwords.


\subsection*{Reproduction}
\addcontentsline{toc}{subsection}{Reproduction}
Because of the security flaws they exploit, the Morris Worm's spreading
mechanisms are particularly noteworthy. It first tries to capitalize on weak
user passwords to create a remote shell with the \textit{rsh} and \textit{rexec}
commands.
If this doesn't work, it uses a buffer overrun in the \textit{finger}
application to create a remote shell on a new machine. Finally, if the two
previous methods are ineffective, the worm sneaks in through a trap door in the
\textit{sendmail} application. These exploits are described in more detail
below.
\subsubsection*{rsh/rexec}
These programs are intended to allow users to spawn shell sessions on remote
machines. After infiltrating a user account as described in the previous
section the worm can use the \textit{rsh} and \textit{rexec} programs to log
into a remote machine and infect it. The worm seeks out new host machines to
infect by looking for connected host accounts in
system tables and user files on an infected machine, or even scanning the
network for other machines.

\subsubsection*{finger}
The finger application gives directory information about users. In 1988 it used
the C library \textit{gets} function to read an incoming request into a buffer.
Even in 1988, this function was known to have a classic buffer overflow
vulnerability, but programmers often prioritize ease of use and convention over security when
writing code. The Morris worm exploited this buffer overflow vulnerability,
writing VAX machine code to create a remote shell into to the buffer and then
overwriting the program's stack frame so that the remote cpu would execute
this code when the request finished processing. This allowed the worm to spread
to any VAX machine running the Berkeley version of the finger server.

\subsubsection*{sendmail}
The sendmail application provides SMTP email service. The Morris Worm exploits
a trap door that is intended to be used for debugging. Entering a debug command
allows one to send commands instead of email to a remote host. The worm uses
this little known feature in an unintended way, sending an anonymous debug email
to a remote host with a copy of the bootstrap program and commands to compile and run
it. This way, the infecting worm doesn't need to open a shell on the remote host
to download and run the bootstrap program.



\subsection*{Maladaptive Traits}
\addcontentsline{toc}{subsection}{Maladaptive Traits}
As witnessed on Black Thursday, the previously described innards of the worm are
keenly adept in their respective purposes, suggesting that the worm program was
fairly well engineered. However, upon further inspection, researchers found
numerous instances of poor programming practice and several critical
bugs that ultimately led to its exposure and subsequent demise.

To avoid infecting the same machine multiple times, the worm 
intermittently checks for other worms on the same machine. If another worm is
found, the two perform a virtual coinflip and the loser agrees to self
terminate.
However, 1 in 7 times, the loser of the coinflip becomes immortal instead of
self terminating, perhaps in order to prevent imposter worms planted by network
administrators from attacking the real worms\cite{spafford_internet_1989}.
Rather than preventing redundant infection, however, this behavior results in
numerous immortal worms clogging up the same machine. This particular bug is
what led to the very noticeable latency and crashing symptoms that affected many
machines on Black Thursday and made the worm rather noticeable.

\cite{spafford_internet_1989}, \cite{seeley_tour_1989}, and \cite{eichin_microscope_1989}
all point many more mistakes, several of which are listed below: 
\begin{itemize}
\item The worm's code doesn't bother to check return conditions for system
calls. This means that it could try to access memory that was
unsuccessfuly allocated, communicate over network connections that failed to
connect, etc. 
\item The worm tries to send a UDP packet the a remote user account,
\textit{ernie.berkeley.edu} over a TCP socket, which, of course, doesn't work.
\item In some places, the code passes incorrect arguments to functions, like a
\textit{struct} instead of a pointer to the \textit{struct}.
\item Several bugs exits in the worm's reproduction exploitations that make it
less effective than it could have been. 
\item If the worm code doesn't compile fast enough on a newly infected system,
error exiting doesn't erase all files, leaving traces of the worm leftover on
disk.
\end{itemize}

Perhaps even more telling than the numerous mistakes in the buggy worm code
are its unused features.


\subsection*{Vestigial Organs}
\addcontentsline{toc}{subsection}{Vestigial Organs}

Researchers dissecting the worm found several
unused attributes. The worm contains redundant code, dead code\footnote{code
that is never used}, and meaningless code. For instance, it tries to remove the
non-existent file \textit{/tmp/.dumb} and it redundantly seeds a random number
generator multiple times. Some of its functions have local variables that are
never used and it contains code that is only called in conditions that can
never be met. 

Dissecters also found conspicuous and curious omissions. The worm doesn't bother
to do anything special with the \textit{root} account if it is cracked, even
though this account gives the worm complete power over the host machine.
It doesn't delete system files, modify other programs, attempt to
\textit{deliberately} disable a system, install back doors or trojan horses,
attempt to record or transmit passwords or other data, or engage in any other
malicious behaviors that one might expect. 

A particularly interesting omission lies in the unused spaces in its file table.
The worm maintains a data structure that can hold information about and transer
up to twenty files, of which it only uses three. These other slots could have
housed files that \textit{do} do the pernicious things
mentioned previously. According to \cite{lee_washpost_2013},

\begin{quote}
An early version of the worm recovered from an automatic backup of Morris's
Cornell files included extensive comments describing Morris's vision for the
project. Those comments suggest that Morris had even more ambitious goals than
he eventually achieved. Morris didn't just want to create a worm that would
silently replicate itself across the Internet. He hoped to build what we would
now call a botnet: a network of thousands of computers coordinating with one
another and available to carry out further instructions at the direction of
their master.

The worm, he wrote in comments on an early version of the worm, will need to
"decide what to break into next" and will need "methods of breaking into other
systems." He also wanted "some way for ME to send out commands, protected by an
encoded password."
\end{quote}

Perhaps it was Morris' intention to transfer the code containing these
capabilities in the empty slots in the file table. \cite{lee_washpost_2013}
speculates that the datagram the worm failed to send to the 
\textit{ernie.berkeley.edu} may have been a beginning phase of the development
of this type of command center. It is clear that had Morris built in this or
any other malicious functionality, the worm could have been much more dangerous. 

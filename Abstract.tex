\section*{Abstract}

On November 2, 1988, a Cornell graduate student named Robert Tappan Morris unleashed one of the first ever computer worms on the fledgling internet. It quickly spread to thousands of connected computers, causing crashes, performance degradation, and panic until it was contained and eradicated. This unprecedented crisis elicited both immediate and long term responses spanning multiple disciplines. The research community responded with the formation of new security emergency diagnosis and response protocols and organizations, including the CERT at CMU. Law enforcement responded by making Robert Morris the first felon convicted under the Computer Fraud and Abuse Act of 1986. The public began to realize the importance of internet security. 

This paper explores the Morris Worm and its overall impact. It details the events surrounding the Morris Worm crisis and the inner workings of the worm itself. It goes on to trace the worm's influence on cyber security legislation, cyber security research, and other cyber attacks over the past several decades. Finally, it draws conclusions about the quality of the worm's overall impact. A jarring and devastating nuisance, the Morris Worm ultimately spread awareness of the gravity and importance of computer security to both legitimate and malevolent users. 
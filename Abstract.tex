\section*{Abstract}

On November $2^{nd}$, 1988, a Cornell graduate student named Robert Tappan
Morris unleashed one of the first ever computer worms on the fledgling internet. It
quickly spread to thousands of connected computers, causing crashes, performance
degradation, and panic until it was contained and eradicated. This unprecedented
crisis elicited both immediate and long term responses spanning multiple
disciplines. The research community responded by creating new security
emergency response protocols and organizations, including the CERT
at CMU. Law enforcement responded by making Robert Morris the first felon
convicted under the Computer Fraud and Abuse Act of 1986. The public at large
began to appreciate the potential impact of computer security, or a lack
thereof.

This paper explores the Morris Worm and its overall impact. It details the
events surrounding the Morris Worm crisis and the inner workings of the worm
itself. It goes on to trace the worm's influence on cyber security legislation,
cyber security research, and other cyber attacks over the past several decades.
Finally, it draws conclusions about the quality of the worm's overall impact. A
jarring and devastating nuisance, the Morris Worm ultimately spread awareness
of computer security to both legitimate and malevolent users.

%%%%%%%%%%%%% OUTBREAK %%%%%%%%%%%%%%%%%%%
\section*{Outbreak}
\addcontentsline{toc}{section}{Outbreak}
This section describes events that took place between November $3^{rd}$ and
Novermber $8^{th}$ of 1988. It outlines the Morris Worm's release, spread, and
eventual eradication at a high level.
Many works covering the Morris Worm begin with some variation of the phrase
``On the evening of November $2^{nd}$, 1988\ldots''
To honor this tradition, the next section does the same. 

%%%%%%%%%%%%% BLACK THURSDAY %%%%%%%%%%%%%%%%%%%
\subsection*{Black Thursday}
\addcontentsline{toc}{subsection}{Black Thursday}

On the evening of Wednesday, November $2^{nd}$ 1988, Robert Tappan Morris, a
first year graduate student at Cornell University, released a worm on the
internet at around 6 pm at the Massachusetts Institute of Technology.
Spreading rapidly, by 11 pm it had infected machines at
the University of Pittsburgh,
RAND Corporation in Santa Monica,
the University of California at Berkely,
the University of Maryland,
the University of Utah,
Stanford,
the University of Minnesota,
and the University of North Carolina
\cite{seeley_tour_1989}
\cite{spafford_internet_1989-1}.

The worm spread faster than even Morris had anticipated. Around
11 pm and again at 11:30 pm, he contacted Andrew Sudduth and Paul Graham
\footnote{
Andrew Sudduth was a friend of Morris'. He and Paul Graham were both
members of the technical staff at Harvard University's Aiken Computational
Laboratory\cite{lee_washpost_2013}
}
to tell them that he had released the worm and to relate steps that could be
taken to stop it. Morris also requested that Andrew Sudduth alert the research
community about the worm and relay how it might be stopped. 
Sudduth did so anonymously via email to a widely used internet research mailing
list, called TCP-IP.
Unfortunately, system administrators had already noticed the worm
and had shut off internet gateways in an effort to quarantine the infection, thereby blocking
Sudduth's email for several days\cite{lee_washpost_2013}\cite{eisenberg_cornell_1989}.

The worm continued to spread throughout the night. By early Thursday morning,
the infection had spread to the University of Arizona, Princeton University,
Lawrence Livermore Labs, UCLA, Purdue University, Georgia Tech, Dartmouth,
the Army Ballistics Research Lab, and the University of Chicago.
By this point, many system administrators were aware of the spreading
infection and Peter Yee of UC Berkeley and NASA Ames Research Center had
posted a message to the TCP-IP mailing list stating ``We are under
attack.''\cite{seeley_tour_1989}\cite{spafford_internet_1989-1} Eventually,
Thursday November $3^{rd}$ would come to be known as ``Black Thursday.'' 

At its peak, the infection is estimated to have spread to around 6,000 machines.
That is 10\% of the approximately 60,000 computers connected to the internet at
that time\cite{eichin_microscope_1989}\cite{marsan_morris_2008}. The worm
targeted systems running 4.2 or 4.3 BSD UNIX and SunOS, causing many of its
infected hosts to crash.
Hosts that didn't crash were riddled with processes that appeared to be shells.
Their process tables, open file tables, and sometimes swap space were exhausted.
Latency soared as legitimate user processes competed with worm processes for cpu
time. Logs showed odd activity that wasn't actually being caused by the
users that they were recording. Finally, strange files were appearing in the
\textit{/usr/tmp} directory\cite{seeley_tour_1989}\cite{spafford_crisis_1989}.

Fortunately, by early Thursday morning the research community had already begun
fighting back.



%%%%%%%%%%%%% INNOCULATION %%%%%%%%%%%%%%%%%%%
\subsection*{Inoculation}
\addcontentsline{toc}{subsection}{Inoculation}

Spafford\cite{spafford_internet_1989-1} points out:
\begin{quote}
It is particularly interesting to note how quickly and how widely the Worm spread.
It is also significant to note how quickly it was identified and stopped by an
ad hoc collection of ``Worm hunters'' using the same network to communicate their
results.
\end{quote}

By 5 am on Thursday, less than 12 hours after the
worm was first discovered, researchers at Berkely had
developed a provisional method to halt its spread\cite{spafford_crisis_1989}.
Black Thursday saw many a researcher and system administrator struggle
with slow machines and trickling network connectivity, as well as 
many swift and staggering successes for them.
They devoloped worm vaccines with impressive speed. 
Mailing lists, including Dr. Spaffords Phage Mailing List
\footnote{
The Phage Mailing List would become the main conduit of communication and source
of information about the worm both during the worm crisis and in subsequent weeks as Morris
was prosecuted
\cite{spafford_crisis_1989}\cite{spafford_internet_1989-1}\cite{lee_washpost_2013}.
}, 
were created to coordinate anti-worm efforts.
Researchers captured the worm and analyzed its behavior. By the end of the day,
software patches had been posted to eliminate the vulnerabilities exploited by
the worm, namely in the \textit{sendmail} and \textit{finger} applications
\cite{spafford_crisis_1989}
\cite{seeley_tour_1989}. By the end of the
day, researchers at Purdue had discovered a
method of stopping the worm without altering
system utilities\cite{spafford_internet_1989-1}. The solution was simply to
create a file called \textit{sh} in the /usr/tmp directory.

Autonamously teaming up, researchers had discovered how to stop the Morris
Worm in just one day. Over the next several
days, the Morris worm was eradicated. ``By Tuesday, November 8, most machines had connected back to the Internet
and traffic patterns had returned to near normal.''\cite{spafford_internet_1989-1}

By early Friday, November $4^{th}$, researchers had disassembled the worm's
code 
\cite{spafford_crisis_1989}
\cite{seeley_tour_1989}.
Several surprises were in store for those performing the autopsy.
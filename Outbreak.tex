\section*{Outbreak}
\addcontentsline{toc}{section}{Outbreak}

\subsection*{Black Thursday}
\addcontentsline{toc}{subsection}{Black Thursday}

Many works covering the Morris Worm begin with some variation of the phrase
``On the evening of November $2^{nd}$, 1988\ldots''.
% \cite{seeley_tour_1989}
% \cite{spafford_crisis_1989}
% \cite{lee_washpost_2013}
% \cite{spafford_internet_1989-1}
% \cite{spafford_internet_1989}
% \cite{eichin_microscope_1989}
This section will do the same: 

On the evening of Wednesday November $2^{nd}$, 1988, Robert Tappan Morris, a
first year graduate student at Cornell University, released a worm onto the
internet. The worm was released at around 6 pm at MIT. Spreading rapidly, by 11
pm it had infected machines at
the University of Pittsburgh,
RAND Corporation in Santa Monica,
UC Berkely,
the University of Maryland,
the University of Utah,
Stanford,
the University of Minnesota,
and the University of North Carolina
\cite{seeley_tour_1989}
\cite{spafford_internet_1989-1}.

Perhaps the worm spread faster than even Morris had anticipated. Around
11 pm and again at around 11:30 pm, he contacted Andrew Sudduth and Paul Graham
\footnote{
Andrew Sudduth was a friend of Morris'. He and Paul Graham were both
members of the technical staff at Harvard University's Aiken Computational
Laboratory \cite{lee_washpost_2013}
}
to tell them that he had released the worm and steps that could be taken to stop
it. Furthermore, Morris requested that Andrew Sudduth alert the research
community of the presence of the worm and how it might be stopped, which
Sudduth did anonymously via email to a widely used internet research mailing
list, called TCP-IP shortly thereafter.
Unfortunately, by that time system administrators had already noticed the worm and had shut off
internet gateways in an effort to quarantine the infection; thereby blocking
Sudduth's email for several days
\cite{lee_washpost_2013}\cite{eisenberg_cornell_1989}.

The worm continued to spread throughout the night. By early Thursday morning,
the infection had spread to the University of Arizona, Princeton University,
Lawrence Livermore Labs, UCLA, Purdue University, Georgia Tech, Dartmouth,
the Army Ballistics Research Lab, and the University of Chicago, amongst others.
By this point many system administrators were aware of the spreading infection, and Peter Yee of UC Berkeley and NASA
Ames Research Center had posted a message to the TCP-IP mailing, stating ``We
are under attack''. Eventually, Thursday November $11^{th}$ would come to be
known as ``Black Thursday''
\cite{seeley_tour_1989}
\cite{spafford_internet_1989-1}. 

At its peak, the infection is estimated to have spread to around 6,000 machines.
That is 10\% of the approximately 60,000 computers connected to the internet at
that time\cite{eichin_microscope_1989}\cite{marsan_morris_2008}. Syptoms of
infection included blah blah blah TODO:*****************

Fortunately, the research community had not only noticed the worm by early
Thursday morning, they had already begun to combat it.


\subsection*{Vaccination}
\addcontentsline{toc}{subsection}{Vaccination}
\begin{itemize}
    \item what researchers did to stop it
    	\begin{itemize}
    	\item sh soln.
        \item software patches
        \item mailing list
        \item quarantined email solutions
        \item disassembly and analysis ***(transition into next section)
    	\end{itemize}
\end{itemize}


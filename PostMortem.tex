\section*{Post Mortem}
\addcontentsline{toc}{section}{Post Mortem}

The Morris Worm helped introduce the public to
perhaps the most revolutionary technological advance in modern times: the
internet. It helped researchers and legislators realize the importance of cyber
security. Subsequent cyber attacks inspired by the Morris Worm followed many of
the same patterns and incited similar responses, encouraging slow growth in
cyber security technology and awareness.

Looking back through the worm hole raises several questions: 

If Morris had not created and released his worm in 1988, what might have turned
out differently? Would no one else have thought to perpetrate a similar attack?
This seems unlikely. As noted previously, researchers were well aware of
security vulnerabilities in the internet, UNIX, and many applications. It was
only a matter of time before something like the Morris Worm would rear its ugly
head and it could have been much worse. The Morris
Worm could have been much more destructive if Robert Morris had spent a little
more time on it. It is unlikely that the folks cleaning up after the worm on
Black Thursday would have thought so, but a look at the worm in hindsight
suggests that we are lucky that events unfolded the way they did.

Does this mean that Morris and other worm programmers are the good guys?
Certainly not. This is analogous to arguing that someone who punched you in the
nose is a good guy for demonstrating that your nose was unprotected. Malware
writers are careless vandals at best and often prove to be fraudulent criminals.
Furthermore, it has been demonstrated that they usually are not very 
technologically adept. Most of the worms mentioned previously contained glaring
flaws resulting from poor programming practices. They exploited vulnerabilities
that were widely known, meaning the attackers didn't even have to discover them
for themselves. Malware writers generally do nothing more than look up how to
punch people in the nose and then use this information to go around punching
people in the nose.
Unfortunately, however, attackers are often skilled enough to elude capture and
prosecution.
Cyber security is a cat and mouse game between attackers and defenders.
Attackers have an easier job.

Would it have been more effective for Morris to warn people about the
vulnerabilities he had discovered rather than release a worm into the wild that
exploits them? Unfortunately, no. As mentioned previously, many system
administrators already knew about all the of vulnerabilities that the Morris
Worm exploited and just refused to do anything about them. In fact, Morris
\textit{had} told the right people about the \textit{finger} bug well over a
year before releasing his worm. It seems that an actual crisis was necessary in
order to get people to start taking security seriously. Even in the years that
followed, system administrators and software vendors ignored warnings from the
computer security community about vulnerable software. Continued lax security
allowed descendants of Morris' Worm and other cyber attacks to flourish. To heed the
warnings of security researchers has been a long, slow lesson and 
we still haven't quite learned it yet, but we are making progress. Security
researchers can now issue warnings from a well recognized platfrom, an option
that Robert Morris didn't have.
